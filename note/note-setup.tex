% note-setup.tex
% 2024-09-21

\usepackage{amsmath,amsthm,amsfonts,amssymb}
\usepackage{mathrsfs}
\usepackage{bm}
\usepackage{mathtools}
\usepackage{float}
\usepackage{appendix}
\usepackage{indentfirst}
\usepackage{silence}
\usepackage{anyfontsize}
\usepackage{xpatch}
\usepackage{extarrows}
\usepackage{booktabs,multirow,multicol}
\usepackage{calligra}
\usepackage{framed}
\usepackage{subcaption}

\usepackage[usenames,dvipsnames]{xcolor}
\usepackage[a4paper, margin=1in]{geometry}

\usepackage{graphicx}
\graphicspath{
    {./figure/}{./figures/}{./image/}{./images/}{./graphic/}{./graphics/}{./picture/}{./pictures/}
}

\usepackage[shortlabels,inline]{enumitem}
\setlist{nolistsep}

\usepackage[ruled,linesnumbered,noline]{algorithm2e}

\usepackage{listings}
\lstdefinestyle{mystyle}{
    basicstyle=\small\ttfamily,
    commentstyle=\color[RGB]{34,139,34},
    keywordstyle=\color[RGB]{0,0,255},
    numberstyle=\tiny\color{gray},
    stringstyle=\color[RGB]{128,0,128},
    identifierstyle=\color{black},
    showstringspaces=false,
    tabsize=4,
    breaklines=true,
    numbers=left,
    frame=single,
    rulecolor=\color{black},
    captionpos=b,
    xleftmargin=\parindent,
    aboveskip=\baselineskip,
    belowskip=\baselineskip,
    escapeinside={\%*}{*)},
}
\lstset{style=mystyle}

\usepackage[colorlinks=true,linkcolor=cyan,urlcolor=magenta,citecolor=violet]{hyperref}

\renewcommand*{\proofname}{\normalfont\bfseries Proof}


\usepackage{thmtools}

\usepackage[framemethod=TikZ]{mdframed}

\mdfsetup{skipabove=1em,skipbelow=0em}


\declaretheoremstyle[
    headfont=\bfseries\sffamily\color{RoyalPurple!70!black}, bodyfont=\normalfont,
    mdframed={
            linewidth=2pt,
            rightline=false, topline=false, bottomline=false,
            linecolor=RoyalPurple, backgroundcolor=RoyalPurple!8,
            nobreak=false
        }
]{thmpurplebox}


\declaretheoremstyle[
    headfont=\bfseries\sffamily\color{NavyBlue!70!black}, bodyfont=\normalfont,
    mdframed={
            linewidth=2pt,
            rightline=false, topline=false, bottomline=false,
            linecolor=NavyBlue, backgroundcolor=NavyBlue!5,
            nobreak=false
        }
]{thmbluebox}

\declaretheoremstyle[
    headfont=\bfseries\sffamily\color{ForestGreen!70!black}, bodyfont=\normalfont,
    mdframed={
            linewidth=2pt,
            rightline=false, topline=false, bottomline=false,
            linecolor=ForestGreen, backgroundcolor=ForestGreen!5,
            nobreak=false
        }
]{thmgreenbox}

\declaretheoremstyle[
    headfont=\bfseries\sffamily\color{RawSienna!70!black}, bodyfont=\normalfont,
    mdframed={
            linewidth=2pt,
            rightline=false, topline=false, bottomline=false,
            linecolor=RawSienna, backgroundcolor=RawSienna!5,
            nobreak=false
        }
]{thmredbox}

\declaretheoremstyle[
    headfont=\bfseries\sffamily\color{WildStrawberry!70!black}, bodyfont=\normalfont,
    mdframed={
            linewidth=2pt,
            rightline=false, topline=false, bottomline=false,
            linecolor=WildStrawberry, backgroundcolor=WildStrawberry!5,
            nobreak=false
        }
]{thmpinkbox}

\declaretheoremstyle[
    headfont=\bfseries\sffamily\color{LimeGreen}, bodyfont=\normalfont,
    mdframed={
            linewidth=2pt,
            rightline=false, topline=false, bottomline=false,
            linecolor=LimeGreen,
            nobreak=false
        }
]{thmgreenline}

\declaretheoremstyle[
    headfont=\bfseries\sffamily\color{TealBlue!70!black}, bodyfont=\normalfont,
    mdframed={
            linewidth=2pt,
            rightline=false, topline=false, bottomline=false,
            linecolor=TealBlue,
            nobreak=false
        }
]{thmblueline}

% usage:
% \newcommand{\noteusemdframed}{true}
% \newcommand{\noteusemdframed}{false}

\ifx\noteusemdframed\undefined
    \newcommand{\noteusemdframed}{false}
\fi

\ifthenelse{\equal{\noteusemdframed}{true}}{
    \declaretheorem[style=thmpurplebox, numbered=yes, name=Theorem, numberwithin=section]{theorem}
    \declaretheorem[style=thmbluebox, name=Corollary, sibling=theorem]{corollary}
    \declaretheorem[style=thmbluebox, name=Lemma, sibling=theorem]{lemma}
    \declaretheorem[style=thmpurplebox, name=Proposition, sibling=theorem]{proposition}

    \declaretheorem[style=thmgreenbox, numbered=yes, name=Definition, numberwithin=section]{definition}
    \declaretheorem[style=thmredbox, name=Example, numberwithin=section]{example}
    \declaretheorem[style=thmpinkbox, name=Problem, numberwithin=section]{problem}

    \declaretheorem[style=thmblueline, numbered=yes, name=Remark, numberwithin=section]{remark}
    \declaretheorem[style=thmblueline, name=Remark, numbered=no]{remark*}
    \declaretheorem[style=thmgreenline, name=Note, numberwithin=section]{note}
    \declaretheorem[style=thmgreenline, name=Note, numbered=no]{note*}
}{
    \declaretheorem[style=plain, numbered=yes, name=Theorem, numberwithin=section]{theorem}
    \declaretheorem[style=plain, name=Corollary, sibling=theorem]{corollary}
    \declaretheorem[style=plain, name=Lemma, sibling=theorem]{lemma}
    \declaretheorem[style=plain, name=Proposition, sibling=theorem]{proposition}

    \declaretheorem[style=definition, numbered=yes, name=Definition, numberwithin=section]{definition}
    \declaretheorem[style=definition, name=Example, numberwithin=section]{example}
    \declaretheorem[style=definition, name=Problem, numberwithin=section]{problem}

    \declaretheorem[style=remark, numbered=yes, name=Remark, numberwithin=section]{remark}
    \declaretheorem[style=remark, name=Remark, numbered=no]{remark*}
    \declaretheorem[style=remark, name=Note, numberwithin=section]{note}
    \declaretheorem[style=remark, name=Note, numbered=no]{note*}
}
