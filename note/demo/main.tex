\documentclass{article}
% note-setup.tex
% 2024-09-21

\usepackage{amsmath,amsthm,amsfonts,amssymb}
\usepackage{mathrsfs}
\usepackage{bm}
\usepackage{mathtools}
\usepackage{float}
\usepackage{appendix}
\usepackage{indentfirst}
\usepackage{silence}
\usepackage{anyfontsize}
\usepackage{xpatch}
\usepackage{extarrows}
\usepackage{booktabs,multirow,multicol}
\usepackage{calligra}
\usepackage{framed}
\usepackage{subcaption}

\usepackage[usenames,dvipsnames]{xcolor}
\usepackage[a4paper, margin=1in]{geometry}

\usepackage{graphicx}
\graphicspath{
    {./figure/}{./figures/}{./image/}{./images/}{./graphic/}{./graphics/}{./picture/}{./pictures/}
}

\usepackage[shortlabels,inline]{enumitem}
\setlist{nolistsep}

\usepackage[ruled,linesnumbered,noline]{algorithm2e}

\usepackage{listings}
\lstdefinestyle{mystyle}{
    basicstyle=\small\ttfamily,
    commentstyle=\color[RGB]{34,139,34},
    keywordstyle=\color[RGB]{0,0,255},
    numberstyle=\tiny\color{gray},
    stringstyle=\color[RGB]{128,0,128},
    identifierstyle=\color{black},
    showstringspaces=false,
    tabsize=4,
    breaklines=true,
    numbers=left,
    frame=single,
    rulecolor=\color{black},
    captionpos=b,
    xleftmargin=\parindent,
    aboveskip=\baselineskip,
    belowskip=\baselineskip,
    escapeinside={\%*}{*)},
}
\lstset{style=mystyle}

\usepackage[colorlinks=true,linkcolor=cyan,urlcolor=magenta,citecolor=violet]{hyperref}

\renewcommand*{\proofname}{\normalfont\bfseries Proof}


\usepackage{thmtools}

\usepackage[framemethod=TikZ]{mdframed}

\mdfsetup{skipabove=1em,skipbelow=0em}


\declaretheoremstyle[
    headfont=\bfseries\sffamily\color{NavyBlue!70!black}, bodyfont=\normalfont,
    mdframed={
            linewidth=2pt,
            rightline=false, topline=false, bottomline=false,
            linecolor=NavyBlue, backgroundcolor=NavyBlue!5,
            nobreak=false
        }
]{thmbluebox}


\declaretheoremstyle[
    headfont=\bfseries\sffamily\color{ForestGreen!70!black}, bodyfont=\normalfont,
    mdframed={
            linewidth=2pt,
            rightline=false, topline=false, bottomline=false,
            linecolor=ForestGreen, backgroundcolor=ForestGreen!5,
            nobreak=false
        }
]{thmgreenbox}

\declaretheoremstyle[
    headfont=\bfseries\sffamily\color{RawSienna!70!black}, bodyfont=\normalfont,
    mdframed={
            linewidth=2pt,
            rightline=false, topline=false, bottomline=false,
            linecolor=RawSienna, backgroundcolor=RawSienna!5,
            nobreak=false
        }
]{thmredbox}

\declaretheoremstyle[
    headfont=\bfseries\sffamily\color{WildStrawberry!70!black}, bodyfont=\normalfont,
    mdframed={
            linewidth=2pt,
            rightline=false, topline=false, bottomline=false,
            linecolor=WildStrawberry, backgroundcolor=WildStrawberry!5,
            nobreak=false
        }
]{thmpinkbox}

\declaretheoremstyle[
    headfont=\bfseries\sffamily\color{ForestGreen}, bodyfont=\normalfont,
    mdframed={
            linewidth=2pt,
            rightline=false, topline=false, bottomline=false,
            linecolor=ForestGreen,
            nobreak=false
        }
]{thmgreenline}

\declaretheoremstyle[
    headfont=\bfseries\sffamily\color{TealBlue!70!black}, bodyfont=\normalfont,
    mdframed={
            linewidth=2pt,
            rightline=false, topline=false, bottomline=false,
            linecolor=TealBlue,
            nobreak=false
        }
]{thmblueline}

% usage:
% \newcommand{\noteusemdframed}{true}
% \newcommand{\noteusemdframed}{false}

\ifx\noteusemdframed\undefined
    \newcommand{\noteusemdframed}{false}
\fi

\ifthenelse{\equal{\noteusemdframed}{true}}{
    \declaretheorem[style=thmbluebox, numbered=yes, name=Theorem, numberwithin=section]{theorem}
    \declaretheorem[style=thmbluebox, name=Corollary, sibling=theorem]{corollary}
    \declaretheorem[style=thmbluebox, name=Lemma, sibling=theorem]{lemma}
    \declaretheorem[style=thmbluebox, name=Proposition, sibling=theorem]{proposition}

    \declaretheorem[style=thmgreenbox, numbered=yes, name=Definition, numberwithin=section]{definition}
    \declaretheorem[style=thmredbox, name=Example, numberwithin=section]{example}
    \declaretheorem[style=thmpinkbox, name=Problem, numberwithin=section]{problem}

    \declaretheorem[style=thmblueline, numbered=yes, name=Remark, numberwithin=section]{remark}
    \declaretheorem[style=thmblueline, name=Remark, numbered=no]{remark*}
    \declaretheorem[style=thmgreenline, name=Note, numberwithin=section]{note}
    \declaretheorem[style=thmgreenline, name=Note, numbered=no]{note*}
}{
    \declaretheorem[style=plain, numbered=yes, name=Theorem, numberwithin=section]{theorem}
    \declaretheorem[style=plain, name=Corollary, sibling=theorem]{corollary}
    \declaretheorem[style=plain, name=Lemma, sibling=theorem]{lemma}
    \declaretheorem[style=plain, name=Proposition, sibling=theorem]{proposition}

    \declaretheorem[style=definition, numbered=yes, name=Definition, numberwithin=section]{definition}
    \declaretheorem[style=definition, name=Example, numberwithin=section]{example}
    \declaretheorem[style=definition, name=Problem, numberwithin=section]{problem}

    \declaretheorem[style=remark, numbered=yes, name=Remark, numberwithin=section]{remark}
    \declaretheorem[style=remark, name=Remark, numbered=no]{remark*}
    \declaretheorem[style=remark, name=Note, numberwithin=section]{note}
    \declaretheorem[style=remark, name=Note, numbered=no]{note*}
}


\title{\LaTeX{} Note Template}
\author{}
\date{}

\begin{document}

\maketitle

\section{Theorem, Proposition, Proof}

\begin{theorem}
    If $1<p<\infty$ and $m > n/p$, or $p=1$ and $m \ge n$, there exist a constant $C = C(m,n,\gamma,p)$, such that
    \begin{equation*}
        \Vert R^m u \Vert_{L^\infty(\Omega)} \le C d^{m-n/p} |u|_{W^m_p(\Omega)}
    \end{equation*}
    for all $u \in W^m_p(\Omega)$.
\end{theorem}
\begin{proof}
    First, we assume that $u \in C^m(\Omega) \cap W^m_p(\Omega)$. We can use the pointwise representation of $R^mu(x)$.
    \begin{align*}
        |R^mu(x)| ={} & m \left| \sum_{|\alpha| = m} \int_{C_x} k_{\alpha}(x,z) D^\alpha u(z)\,dz \right| \notag \\
        \le{}         & C \sum_{|\alpha|=m} \int_{\Omega} |x-z|^{-n+m} |D^\alpha u(z)|\,dz \notag                \\
        \le{}         & C' d^{m-n/p} |u|_{W^m_p(\Omega)}.
    \end{align*}
    The proof can be completed via a density argument.
\end{proof}

\begin{theorem}[xxx]
    If $1<p<\infty$ and $m > n/p$, or $p=1$ and $m \ge n$, there exist a constant $C = C(m,n,\gamma,p)$, such that
    \begin{equation*}
        \Vert R^m u \Vert_{L^\infty(\Omega)} \le C d^{m-n/p} |u|_{W^m_p(\Omega)}
    \end{equation*}
    for all $u \in W^m_p(\Omega)$.
\end{theorem}
\begin{proof}[\upshape\bfseries Proof of xxx]
    First, we assume that $u \in C^m(\Omega) \cap W^m_p(\Omega)$. We can use the pointwise representation of $R^mu(x)$.
    \begin{align*}
        |R^mu(x)| ={} & m \left| \sum_{|\alpha| = m} \int_{C_x} k_{\alpha}(x,z) D^\alpha u(z)\,dz \right| \notag \\
        \le{}         & C \sum_{|\alpha|=m} \int_{\Omega} |x-z|^{-n+m} |D^\alpha u(z)|\,dz \notag                \\
        \le{}         & C' d^{m-n/p} |u|_{W^m_p(\Omega)}.
    \end{align*}
    The proof can be completed via a density argument.
\end{proof}

\begin{theorem*}
    If $1<p<\infty$ and $m > n/p$, or $p=1$ and $m \ge n$, there exist a constant $C = C(m,n,\gamma,p)$, such that
    \begin{equation*}
        \Vert R^m u \Vert_{L^\infty(\Omega)} \le C d^{m-n/p} |u|_{W^m_p(\Omega)}
    \end{equation*}
    for all $u \in W^m_p(\Omega)$.
\end{theorem*}
\begin{proof}
    First, we assume that $u \in C^m(\Omega) \cap W^m_p(\Omega)$. We can use the pointwise representation of $R^mu(x)$.
    \begin{align*}
        |R^mu(x)| ={} & m \left| \sum_{|\alpha| = m} \int_{C_x} k_{\alpha}(x,z) D^\alpha u(z)\,dz \right| \notag \\
        \le{}         & C \sum_{|\alpha|=m} \int_{\Omega} |x-z|^{-n+m} |D^\alpha u(z)|\,dz \notag                \\
        \le{}         & C' d^{m-n/p} |u|_{W^m_p(\Omega)}.
    \end{align*}
    The proof can be completed via a density argument.
\end{proof}

\begin{proposition}
    \begin{equation*}
        Q^m u(x) = \sum_{|\lambda| < m} \left( \int_B \psi_\lambda(y) u(y)\,dy \right) x^\lambda
    \end{equation*}
    where $\psi_\lambda \in C_0^\infty(\mathbb{R}^n)$ and $\mathrm{supp}(\phi_\lambda) \in \overline{B}$.
\end{proposition}
\begin{proof}
    This follows from xxx if we define
    \begin{equation*}
        \psi_\lambda(y) = \sum_{\alpha \ge \lambda,|\alpha|<m}
        \frac{(-1)^{|\alpha|}}{\alpha !} a_{[\lambda,\alpha-\lambda]} D^\alpha(y^{\alpha-\lambda} \phi(y)).
    \end{equation*}
\end{proof}

\begin{proposition}[xxx]
    \begin{equation*}
        Q^m u(x) = \sum_{|\lambda| < m} \left( \int_B \psi_\lambda(y) u(y)\,dy \right) x^\lambda
    \end{equation*}
    where $\psi_\lambda \in C_0^\infty(\mathbb{R}^n)$ and $\mathrm{supp}(\phi_\lambda) \in \overline{B}$.
\end{proposition}
\begin{proof}[\upshape\bfseries Proof of xxx]
    This follows from xxx if we define
    \begin{equation*}
        \psi_\lambda(y) = \sum_{\alpha \ge \lambda,|\alpha|<m}
        \frac{(-1)^{|\alpha|}}{\alpha !} a_{[\lambda,\alpha-\lambda]} D^\alpha(y^{\alpha-\lambda} \phi(y)).
    \end{equation*}
\end{proof}

\begin{proposition*}
    \begin{equation*}
        Q^m u(x) = \sum_{|\lambda| < m} \left( \int_B \psi_\lambda(y) u(y)\,dy \right) x^\lambda
    \end{equation*}
    where $\psi_\lambda \in C_0^\infty(\mathbb{R}^n)$ and $\mathrm{supp}(\phi_\lambda) \in \overline{B}$.
\end{proposition*}
\begin{proof}
    This follows from xxx if we define
    \begin{equation*}
        \psi_\lambda(y) = \sum_{\alpha \ge \lambda,|\alpha|<m}
        \frac{(-1)^{|\alpha|}}{\alpha !} a_{[\lambda,\alpha-\lambda]} D^\alpha(y^{\alpha-\lambda} \phi(y)).
    \end{equation*}
\end{proof}

\section{Corollary, Lemma, Claim}

\begin{corollary}
    Under the assumption of xxx, the following inequality holds
    \begin{equation*}
        \inf_{v \in P^{m-1}} \Vert u - v \Vert_{W^k_p(\Omega)} \le C_{m,n,\gamma} d^{m-k} |u|_{W^k_p(\Omega)}, \,\, k = 0,1,\dots,m,
    \end{equation*}
\end{corollary}

\begin{corollary}[xxx]
    Under the assumption of xxx, the following inequality holds
    \begin{equation*}
        \inf_{v \in P^{m-1}} \Vert u - v \Vert_{W^k_p(\Omega)} \le C_{m,n,\gamma} d^{m-k} |u|_{W^k_p(\Omega)}, \,\, k = 0,1,\dots,m,
    \end{equation*}
\end{corollary}

\begin{corollary*}
    Under the assumption of xxx, the following inequality holds
    \begin{equation*}
        \inf_{v \in P^{m-1}} \Vert u - v \Vert_{W^k_p(\Omega)} \le C_{m,n,\gamma} d^{m-k} |u|_{W^k_p(\Omega)}, \,\, k = 0,1,\dots,m,
    \end{equation*}
\end{corollary*}

\begin{lemma}
    Let $f \in L^p(\Omega)$ for $p \ge 1$ and $m \ge 1$ and let
    \[
        g(x) = \int_\Omega |x-z|^{-n+m} |f(z)|\,dz
    \]
    Then
    \begin{equation*}
        \Vert g \Vert_{L^p(\Omega)} \le C_{m,n} d^m \Vert f\Vert_{L^p(\Omega)}.
    \end{equation*}
\end{lemma}

\begin{lemma}[xxx]
    Let $f \in L^p(\Omega)$ for $p \ge 1$ and $m \ge 1$ and let
    \[
        g(x) = \int_\Omega |x-z|^{-n+m} |f(z)|\,dz
    \]
    Then
    \begin{equation*}
        \Vert g \Vert_{L^p(\Omega)} \le C_{m,n} d^m \Vert f\Vert_{L^p(\Omega)}.
    \end{equation*}
\end{lemma}

\begin{lemma*}
    Let $f \in L^p(\Omega)$ for $p \ge 1$ and $m \ge 1$ and let
    \[
        g(x) = \int_\Omega |x-z|^{-n+m} |f(z)|\,dz
    \]
    Then
    \begin{equation*}
        \Vert g \Vert_{L^p(\Omega)} \le C_{m,n} d^m \Vert f\Vert_{L^p(\Omega)}.
    \end{equation*}
\end{lemma*}

\begin{claim}
    $Q^m u$ is a polynomial of degree less than $m$ in $x$.
\end{claim}

\begin{claim}[xxx]
    $Q^m u$ is a polynomial of degree less than $m$ in $x$.
\end{claim}

\begin{claim*}
    $Q^m u$ is a polynomial of degree less than $m$ in $x$.
\end{claim*}

\section{Definition}

\begin{definition}
    $\Omega$ is star-shaped with respect to the ball $B$ if , for all $x \in \Omega$, the closed convex hull of $\{x\} \cup B$ is a subset of $\Omega$.
\end{definition}

\begin{definition}[xxx]
    $\Omega$ is star-shaped with respect to the ball $B$ if , for all $x \in \Omega$, the closed convex hull of $\{x\} \cup B$ is a subset of $\Omega$.
\end{definition}

\begin{definition*}
    $\Omega$ is star-shaped with respect to the ball $B$ if , for all $x \in \Omega$, the closed convex hull of $\{x\} \cup B$ is a subset of $\Omega$.
\end{definition*}

\section{Example}

\begin{example}
    The integral form of the Taylor remainder for $f \in C^m([0,1])$ is given by
    \begin{equation*}
        f(s) = \sum_{k=0}^{m-1}\frac{1}{k!} f^{(k)}(0) + \int_0^s \frac{1}{(m-1)!} f^{(m)}(t)(s-t)^{m-1}\,dt.
    \end{equation*}
\end{example}

\begin{example}[xxx]
    The integral form of the Taylor remainder for $f \in C^m([0,1])$ is given by
    \begin{equation*}
        f(s) = \sum_{k=0}^{m-1}\frac{1}{k!} f^{(k)}(0) + \int_0^s \frac{1}{(m-1)!} f^{(m)}(t)(s-t)^{m-1}\,dt.
    \end{equation*}
\end{example}

\begin{example*}
    The integral form of the Taylor remainder for $f \in C^m([0,1])$ is given by
    \begin{equation*}
        f(s) = \sum_{k=0}^{m-1}\frac{1}{k!} f^{(k)}(0) + \int_0^s \frac{1}{(m-1)!} f^{(m)}(t)(s-t)^{m-1}\,dt.
    \end{equation*}
\end{example*}

\section{Problem, Solution}

\begin{problem}
Calculate the integral of the function $g(x) = 3x^2$ with respect to $x$.
\end{problem}

\begin{solution}
    To calculate the integral of $g(x) = 3x^2$, we use the power rule for integration:
    \[
        \int 3x^2 \, dx = x^3 + C
    \]
    where $C$ is the constant of integration.
\end{solution}

\begin{problem}[xxx]
Calculate the integral of the function $g(x) = 3x^2$ with respect to $x$.
\end{problem}
\begin{solution}[xxx]
    To calculate the integral of $g(x) = 3x^2$, we use the power rule for integration:
    \[
        \int 3x^2 \, dx = x^3 + C
    \]
    where $C$ is the constant of integration.
\end{solution}

\begin{problem*}
    Calculate the integral of the function $g(x) = 3x^2$ with respect to $x$.
\end{problem*}
\begin{solution*}
    To calculate the integral of $g(x) = 3x^2$, we use the power rule for integration:
    \[
        \int 3x^2 \, dx = x^3 + C
    \]
    where $C$ is the constant of integration.
\end{solution*}

\section{Remark}

\begin{remark}
    Such a polynomial is not unique, due to the choice od cut-off function $\phi$.
\end{remark}

\begin{remark}[xxx]
    Such a polynomial is not unique, due to the choice od cut-off function $\phi$.
\end{remark}

\begin{remark*}
    Such a polynomial is not unique, due to the choice od cut-off function $\phi$.
\end{remark*}

\section{Note}

\begin{note}
    The degree of $Q^m u$ is at most $m-1$.
\end{note}

\begin{note}[xxx]
    The degree of $Q^m u$ is at most $m-1$.
\end{note}

\begin{note*}
    The degree of $Q^m u$ is at most $m-1$.
\end{note*}


\section{lstlisting}

\begin{lstlisting}[language=Python,caption={hello world}]
def hello():
    print("Hello, world!")

hello()
\end{lstlisting}

\begin{lstlisting}[language=Python,caption={hanoi.py}]
step = 1


def hanoi(n, a, b, c, depth=0):
    def move(n, a, c):
        global step
        print("    " * depth, end="")
        print(f"{step=}: move [{n}] from {a} to {c}")
        step += 1

    if n == 1:
        move(n, a, c)
    else:
        hanoi(n - 1, a, c, b, depth=depth + 1)
        move(n, a, c)
        hanoi(n - 1, b, a, c, depth=depth + 1)


if __name__ == "__main__":
    n = int(input("Hanoi Problem, N = "))
    hanoi(n, "A", "B", "C")
\end{lstlisting}

\section{algorithm}


\begin{algorithm}[H]
    \KwIn{This is some input}
    \KwOut{This is some output}
    \SetAlgoLined
    \SetNoFillComment
    \tcc{This is a comment}
    \vspace{3mm}
    some code here\;
    $x \leftarrow 0$\;
    $y \leftarrow 0$\;
    \uIf{$ x > 5$} {
        x is greater than 5 \tcp*{This is also a comment}
    }
    \Else {
        x is less than or equal to 5\;
    }
    \ForEach{y in 0..5} {
        $y \leftarrow y + 1$\;
    }
    \For{$y$ in $0..5$} {
        $y \leftarrow y - 1$\;
    }
    \While{$x > 5$} {
        $x \leftarrow x - 1$\;
    }
    \Return Return something here\;
    \caption{what}
\end{algorithm}

\end{document}
