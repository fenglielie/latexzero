\documentclass{article}
%================================
% note-setup-box.tex
% fenglielie@qq.com 2025-05-06
%================================

\usepackage{amsmath,amsthm,amsfonts,amssymb}
\usepackage{mathrsfs}
\usepackage{bm}
\usepackage{mathtools}
\usepackage{float}
\usepackage{appendix}
\usepackage{indentfirst}
\usepackage{silence}
\usepackage{anyfontsize}
\usepackage{xpatch}
\usepackage{extarrows}
\usepackage{booktabs,multirow,multicol}
\usepackage{calligra}
\usepackage{subcaption}

\usepackage[dvipsnames]{xcolor}
\usepackage[a4paper, margin=1in]{geometry}

\usepackage{graphicx}
\graphicspath{
    {./figure/}{./figures/}{./image/}{./images/}{./graphic/}{./graphics/}{./picture/}{./pictures/}
}

\usepackage[shortlabels,inline]{enumitem}
\setlist{nolistsep}

\usepackage[ruled,linesnumbered,noline]{algorithm2e}

\usepackage{listings}
\lstdefinestyle{simpleStyle}{
    basicstyle=\ttfamily\small,
    breaklines=true,
    keywordstyle=\color{blue},
    identifierstyle=\color{black},
    stringstyle=\color{violet},
    commentstyle=\color[RGB]{34,139,34},
    showstringspaces=false,
    numbers=left,
    numbersep=2em,
    numberstyle=\footnotesize,
    frame=single,
    framesep=1em,
}
\lstset{style=simpleStyle}

\usepackage[colorlinks=true,linkcolor=cyan,urlcolor=magenta,citecolor=violet]{hyperref}

\renewcommand*{\proofname}{\normalfont\bfseries Proof}

\usepackage{thmtools}
\usepackage[most]{tcolorbox}
\tcbuselibrary{theorems}

\newcommand{\bindtcbtheorem}[2]{%
    \tcolorboxenvironment{#1}{#2}
    \tcolorboxenvironment{#1*}{#2}
}

% Purple half box: Theorem, Proposition

\bindtcbtheorem{theorem}{boxrule=1pt, colframe=RoyalPurple, colback=RoyalPurple!8, enhanced, breakable}

\declaretheorem[style=plain, name=Theorem, numbered=yes, numberwithin=section]{theorem}
\declaretheorem[style=plain, name=Theorem, numbered=no]{theorem*}

\bindtcbtheorem{proposition}{boxrule=1pt, colframe=RoyalPurple, colback=RoyalPurple!8, enhanced, breakable}

\declaretheorem[style=plain, name=Proposition, numbered=yes, sibling=theorem]{proposition}
\declaretheorem[style=plain, name=Proposition, numbered=no]{proposition*}

% Blue half box: Corollary, Lemma, Claim

\bindtcbtheorem{corollary}{boxrule=1pt, colframe=NavyBlue, colback=SkyBlue!8, enhanced, breakable}

\declaretheorem[style=plain, name=Corollary, numbered=yes, sibling=theorem]{corollary}
\declaretheorem[style=plain, name=Corollary, numbered=no]{corollary*}

\bindtcbtheorem{lemma}{boxrule=1pt, colframe=NavyBlue, colback=SkyBlue!8, enhanced, breakable}

\declaretheorem[style=plain, name=Lemma, numbered=yes, sibling=theorem]{lemma}
\declaretheorem[style=plain, name=Lemma, numbered=no]{lemma*}

\bindtcbtheorem{claim}{boxrule=1pt, colframe=NavyBlue, colback=SkyBlue!8, enhanced, breakable}

\declaretheorem[style=plain, name=Claim, numbered=yes, sibling=theorem]{claim}
\declaretheorem[style=plain, name=Claim, numbered=no]{claim*}

% Green full box: Definition
\bindtcbtheorem{definition}{boxrule=1pt, colframe=ForestGreen, colback=ForestGreen!5, enhanced, breakable}

\declaretheorem[style=definition, name=Definition, numbered=yes, numberwithin=section]{definition}
\declaretheorem[style=definition, name=Definition, numbered=no]{definition*}

% Red full box: Example
\bindtcbtheorem{example}{boxrule=1pt, colframe=RawSienna, colback=RawSienna!5, enhanced, breakable}

\declaretheorem[style=definition, name=Example, numbered=yes, numberwithin=section]{example}
\declaretheorem[style=definition, name=Example, numbered=no]{example*}

% Pink empty box: Problem
\bindtcbtheorem{problem}{boxrule=0pt, colback=WildStrawberry!5, enhanced, breakable}

\declaretheorem[style=definition, name=Problem, numbered=yes, numberwithin=section]{problem}
\declaretheorem[style=definition, name=Problem, numbered=no]{problem*}

% Brown empty box: Remark
\bindtcbtheorem{remark}{boxrule=0pt, colback=white, enhanced, breakable, frame hidden}

\declaretheorem[style=remark, name=Remark, numbered=yes, numberwithin=section]{remark}
\declaretheorem[style=remark, name=Remark, numbered=no]{remark*}

% Green empty box: Note
\bindtcbtheorem{note}{boxrule=0pt, colback=white, enhanced, breakable, frame hidden}

\declaretheorem[style=remark, name=Note, numbered=yes, numberwithin=section]{note}
\declaretheorem[style=remark, name=Note, numbered=no]{note*}

\declaretheoremstyle[headfont=\bfseries, bodyfont=\normalfont, spaceabove=6pt, spacebelow=6pt, qed=\ensuremath{\square}]{solutionstyle}

\declaretheorem[style=solutionstyle, name=Solution, numbered=yes, numberwithin=section]{solution}
\declaretheorem[style=solutionstyle, name=Solution, numbered=no]{solution*}


\usepackage[style=numeric]{biblatex}
\IfFileExists{reference.bib}{\addbibresource{reference.bib}}{}

\title{Note on Polynomial Approximation}
\author{Author}
\date{\today}

\begin{document}

\maketitle

\section{Averaged Taylor Polynomial}

\begin{definition}
    The Taylor polynomial of order $m$ evaulated at $y$ is given by
    \begin{equation}\label{eq:taylor}
        T_y^mu(x) = \sum_{|\alpha| < m} \frac{1}{\alpha !} D^\alpha u(y) (x-y)^\alpha
    \end{equation}
    where $\alpha = (\alpha_1,\dots,\alpha_n)$ is an n-tuple of non-negative integers, $x \in \mathbb{R}^n$ and
    \[
        x^{\alpha} = \prod_{i=1}^n x_i^{\alpha_i},\,
        \alpha ! = \prod_{i=1}^n \alpha_i!,\,
        \vert \alpha \vert = \sum_{i=1}^n \alpha_i
    \]
\end{definition}

Howerer, if $u$ is in a Sobolev space, $D^\alpha u$ may not exist in the usual pointwise sense.
We accomplish this by taking an ``average'' over $y$ on a ball.


\begin{definition}
    Suppose $u$ has weak derivatives of order strictly less than $m$ in a region $\Omega$ such than $B \subset\subset \Omega$.
    The corresponding Taylor polynomial of order $m$ of $u$ averaged over $B$ is defined as
    \begin{equation}\label{eq:avg_taylor}
        Q^m u(x) = \int_B T_y^m u(x) \phi(y)\,dy
    \end{equation}
    where $T_y^m u(x)$ is defined as \eqref{eq:taylor}, $B$ is a ball centerred at $x_0$ with radius $\rho$ and $\phi$ is the cut-off function supported in $\overline{B}$.
\end{definition}

\begin{remark}
    Such a polynomial is not unique, due to the choice od cut-off function $\phi$.
\end{remark}

\begin{proposition}
    $Q^m u$ is a polynomial of degree less than $m$ in $x$.
\end{proposition}
\begin{proof}
    The definition of $Q^m u$ does make sence because $u \in W_p^{m-1}(\Omega)$.
    If we write
    \[
        (x-y)^\alpha = \prod_{i=1}^n (x_i-y_i)^{\alpha_i} = \sum_{\gamma+\beta = \alpha} a_{[\gamma,\beta]} x^{\gamma} y^{\beta}
    \]
    where $\gamma = (\gamma_1,\dots,\gamma_n)$, $\beta = (\beta_1,\dots,\beta_n)$ and $a_{[\gamma,\beta]}$ are constants,
    $Q^m u(x)$ can be written as
    \begin{align*}
        Q^m u(x) & = \int_B T_y^m u(x) \phi(y)\,dy = \sum_{|\alpha|<m} \int_B \frac{1}{\alpha !} D^\alpha u(y) (x-y)^\alpha \phi(y)\,dy
        \\
                 & = \sum_{|\alpha|<m}  \sum_{\gamma+\beta = \alpha}
        \left(\frac{1}{\alpha !} a_{[\gamma,\beta]}  \int_B D^\alpha u(y) y^{\beta} \phi(y)\,dy \right) x^{\gamma}
    \end{align*}
\end{proof}

\begin{note}
    The degree of $Q^m u$ is at most $m-1$.
\end{note}


Though we use the assumption that $u$ is in a Sobolev space $W_p^{m-1}(\Omega)$,
we can extend the definition of $Q^m u$ to $L^1(B)$ by integrating by parts:
\begin{equation}\label{eq:avg_taylor2}
    Q^m u(x) = \sum_{|\alpha|<m}  \sum_{\gamma+\beta = \alpha}
    \left(\frac{(-1)^{|\alpha|}}{\alpha !} a_{[\gamma,\beta]}  \int_B u(y) D^\alpha(y^{\beta} \phi(y))\,dy \right) x^{\gamma}.
\end{equation}
It is equivalent to \eqref{eq:avg_taylor} if $u \in W_p^{m-1}(\Omega)$.

\begin{proposition}
    \begin{equation} \label{eq:avg_taylor3}
        Q^m u(x) = \sum_{|\lambda| < m} \left( \int_B \psi_\lambda(y) u(y)\,dy \right) x^\lambda
    \end{equation}
    where $\psi_\lambda \in C_0^\infty(\mathbb{R}^n)$ and $\mathrm{supp}(\phi_\lambda) \in \overline{B}$.
\end{proposition}

\begin{proof}
    This follows from \eqref{eq:avg_taylor2} if we define
    \begin{equation}
        \psi_\lambda(y) = \sum_{\alpha \ge \lambda,|\alpha|<m}
        \frac{(-1)^{|\alpha|}}{\alpha !} a_{[\lambda,\alpha-\lambda]} D^\alpha(y^{\alpha-\lambda} \phi(y)).
    \end{equation}
\end{proof}

\begin{corollary}
    If $\Omega$ is a bounded domain in $\mathbb{R}^n$, then for any $k$, $Q^m$ is a bounded map of $L^1(B)$ into $W^{k}_\infty(\Omega)$.
    There exist a constant $C = C(m,n,\rho,\Omega)$, such that for all $u \in L^1(B)$,
    \begin{equation}
        \Vert Q^m u \Vert_{W^{k}_\infty(\Omega)} \le C \Vert u \Vert_{L^1(B)}.
    \end{equation}
\end{corollary}
\begin{proof}
    This follows from \eqref{eq:avg_taylor3} and the fact that both $\sup_{y \in B} |\psi_\lambda(y)|$ and $\sup_{x \in \Omega} |D^\alpha x^\lambda |$ are bounded.
\end{proof}

\begin{proposition}
    For any $\alpha$ such than $|\alpha| \le m-1$,
    \begin{equation}
        D^\alpha Q^m u = Q^{m - |\alpha|} D^\alpha u, \forall u \in W_1^{|\alpha|}(B).
    \end{equation}
\end{proposition}
\begin{proof}
    It's easy to verify this proposition for all $u \in C^\infty(\Omega)$.
    The proof is completed via a density argument.
\end{proof}

\section{Error Representation}

\begin{definition}
    $\Omega$ is star-shaped with respect to the ball $B$ if , for all $x \in \Omega$, the closed convex hull of $\{x\} \cup B$ is a subset of $\Omega$.
\end{definition}

From now on, we assume that $\Omega$ is star-shaped with respect to the ball $B$. Let $C_x$ denote the convex hull of $\{x\} \cup B$.

The integral form of the Taylor remainder for $f \in C^m([0,1])$ is given by
\begin{equation}
    f(s) = \sum_{k=0}^{m-1}\frac{1}{k!} f^{(k)}(0) + \int_0^s \frac{1}{(m-1)!} f^{(m)}(t)(s-t)^{m-1}\,dt.
\end{equation}
Let $u$ be a $C^m$ function on $\Omega$. For $x \in \Omega$ and $y \in B$, define $f(s) = u(y + s(x-y))$.
Then, we obtain
\begin{equation}
    \frac{1}{k!} f^{(k)}(s) = \sum_{|\alpha| = k}\frac{1}{\alpha !} D^\alpha u(y + s(x-y)) (x-y)^{\alpha}.
\end{equation}
Hence,
\begin{align}
    f(s) ={} & \sum_{k=0}^{m-1} \sum_{|\alpha| = k}\frac{1}{\alpha !} D^\alpha u(y) (x-y)^{\alpha} \notag                             \\
             & + m \int_0^s \left(\sum_{|\alpha| = m}\frac{1}{\alpha !} D^\alpha u(y + t(x-y)) (x-y)^{\alpha} \right)(s-t)^{m-1}\,dt.
\end{align}
Take $s=1$, we have $f(1) = u(x)$. By using the variable substitution $t \to 1-t$ in the integral, we obtain
\begin{align}
    u(x) ={} & \sum_{|\alpha| < m}\frac{1}{\alpha !} D^\alpha u(y) (x-y)^{\alpha}  \notag                                         \\
             & + m \int_0^1 \left(\sum_{|\alpha| = m}\frac{1}{\alpha !} D^\alpha u(x + t(y-x)) (x-y)^{\alpha} \right)t^{m-1}\,dt.
    \\
    ={}      & \sum_{|\alpha| < m}\frac{1}{\alpha !} D^\alpha u(y) (x-y)^{\alpha} \notag                                          \\
             & + m \sum_{|\alpha| = m} (x-y)^{\alpha}\int_0^1 \frac{1}{\alpha !} D^\alpha u(x + t(y-x)) t^{m-1}\,dt.
\end{align}

\begin{definition}
    The $m$-th order renaubder term of $Q^m u$ is given by
    \[
        R^m u(x) = u(x) - Q^m u(x)
    \]
\end{definition}

Using the definition of $Q^m$ and cut-off function, we obtain
\begin{align}
    R^m u(x) ={} & u(x) - Q^m u(x) \notag                                                                                                              \\
    ={}          & \int_B u(x) \phi(y)\,dy - \int_B T_y^m u(x) \phi(y)\,dy \notag                                                                      \\
    ={}          & \int_B \left(u(x) - T_y^m u(x)\right) \phi(y)\,dy \notag                                                                            \\
    ={}          & \int_B \left(m \sum_{|\alpha| = m} (x-y)^{\alpha}\int_0^1 \frac{1}{\alpha !} D^\alpha u(x + t(y-x)) t^{m-1}\,dt\right) \phi(y)\,dy.
\end{align}

\begin{proposition}\label{prop:avg_remainder}
    The remainder $R^m u$ satisfies
    \begin{equation}
        R^m u(x) = m \sum_{|\alpha| = m} \int_{C_x} k_{\alpha}(x,z) D^\alpha u(z)\,dz
    \end{equation}
    where $z = x + t(y-z)$, $k_{\alpha}(x,z) = \frac{1}{\alpha !}(x-z)^{\alpha} k(x,z)$ and
    \begin{equation}
        \vert k(x,z) \vert \le C \left( 1 + \frac{1}{\rho}|x-x_0|\right)^n |x-z|^{-n}
    \end{equation}
\end{proposition}
The proof is omitted.


\begin{definition}
    Suppose $\Omega$ has diameter $d$ and is star-shaped with respect to a ball $B$.
    Then the chunkiness parameter of $\Omega$ is defined as
    \[
        \gamma = \frac{d}{\rho_{\max}},
    \]
    where $\rho_{\max}$ is the largest radius of all ball $B$,
    \[
        \rho_{\max} = \sup \{\rho \mid \text{$\Omega$ is star-shaped with respect to a ball of radius $\rho$}\}
    \]
\end{definition}

\begin{corollary}
    The ball $B$ can be chosen so that the function $k(x,z)$ in proposition \ref{prop:avg_remainder}
    satisfies the following estimate:
    \begin{equation}
        |k(x,z)| \le C(1+\gamma)^n |x-z|^{-n}, \forall x \in \Omega,
    \end{equation}
    where $\gamma$ is the chunkiness parameter of $\Omega$.
\end{corollary}
\begin{proof}
    Choose a ball $B$ such that $\Omega$ is star-shaped with respect to $B$ and the radius of $B$ satisfies $\rho \ge \frac12 \rho_{\max}$.
    Then
    \begin{align}
        \vert k(x,z) \vert \le{} & C \left( 1 + \frac{1}{\rho}|x-x_0|\right)^n |x-z|^{-n} \notag  \\
        \le{}                    & C \left( 1 + \frac{2d}{\rho_{\max}}\right)^n |x-z|^{-n} \notag \\
        \le{}                    & 2^n C (1+\gamma)^n |x-z|^{-n}.
    \end{align}
\end{proof}


\section{Bounds for Riesz Potentials}

\begin{lemma}
    If $f \in L^p(\Omega)$ for $1<p<\infty$ and $m > n/p$, then
    \begin{equation}
        \int_{\Omega} |x-z|^{-n+m} |f(z)|\,dz \le C_p d^{m-n/p} \Vert f \Vert_{L^p(\Omega)}, \forall x \in \Omega.
    \end{equation}
    This inequality also holds for $p=1$ if $m \ge n$.
\end{lemma}

\begin{proposition}
    If $1<p<\infty$ and $m > n/p$, or $p=1$ and $m \ge n$, there exist a constant $C = C(m,n,\gamma,p)$, such that
    \begin{equation}
        \Vert R^m u \Vert_{L^\infty(\Omega)} \le C d^{m-n/p} |u|_{W^m_p(\Omega)}
    \end{equation}
    for all $u \in W^m_p(\Omega)$.
\end{proposition}
\begin{proof}
    First, we assume that $u \in C^m(\Omega) \cap W^m_p(\Omega)$. We can use the pointwise representation of $R^mu(x)$.
    \begin{align}
        |R^mu(x)| ={} & m \left| \sum_{|\alpha| = m} \int_{C_x} k_{\alpha}(x,z) D^\alpha u(z)\,dz \right| \notag \\
        \le{}         & C \sum_{|\alpha|=m} \int_{\Omega} |x-z|^{-n+m} |D^\alpha u(z)|\,dz \notag                \\
        \le{}         & C' d^{m-n/p} |u|_{W^m_p(\Omega)}.
    \end{align}
    The proof can be completed via a density argument.
\end{proof}

\begin{lemma}
    Let $f \in L^p(\Omega)$ for $p \ge 1$ and $m \ge 1$ and let
    \[
        g(x) = \int_\Omega |x-z|^{-n+m} |f(z)|\,dz
    \]
    Then
    \begin{equation}
        \Vert g \Vert_{L^p(\Omega)} \le C_{m,n} d^m \Vert f\Vert_{L^p(\Omega)}.
    \end{equation}
\end{lemma}

\begin{lemma}[Bramble-Hilbert] \label{lemma:bramble-hilbert}
    Let $B$ be a ball in $\Omega$ such that $\Omega$ is star-shaped with respect to $B$ and such that its radisu
    $\rho > \frac 12 \rho_{\max}$. Let $Q^m u$ be the Taylor polynomial of order $m$ of $u$ averaged over $B$ where
    $u \in W_p^m(\Omega)$ and $p \ge 1$. Then
    \begin{equation}
        \vert u - Q^m u \vert_{W^k_p(\Omega)} \le C_{m,n,\gamma} d^{m-k} |u|_{W^k_p(\Omega)}, \,\, k = 0,1,\dots,m,
    \end{equation}
    where $d = \mathrm{diam}(\Omega)$.
\end{lemma}

\begin{note}
    Bramble-Hilbert lemma is an important result for the analysis of the approximation properties of finite elements.
\end{note}

\begin{corollary}
    Under the assumption of the \autoref{lemma:bramble-hilbert}, the following inequality holds
    \begin{equation}
        \inf_{v \in P^{m-1}} \Vert u - v \Vert_{W^k_p(\Omega)} \le C_{m,n,\gamma} d^{m-k} |u|_{W^k_p(\Omega)}, \,\, k = 0,1,\dots,m,
    \end{equation}
\end{corollary}
\begin{proof}
    \autoref{lemma:bramble-hilbert} provides a constructive proof.
\end{proof}

\begin{lstlisting}[language=Python]
# Python
def hello():
    print("Hello, world!")

hello()
\end{lstlisting}

\begin{algorithm}[H]
    \KwIn{This is some input}
    \KwOut{This is some output}
    \SetAlgoLined
    \SetNoFillComment
    \tcc{This is a comment}
    \vspace{3mm}
    some code here\;
    $x \leftarrow 0$\;
    $y \leftarrow 0$\;
    \uIf{$ x > 5$} {
        x is greater than 5 \tcp*{This is also a comment}
    }
    \Else {
        x is less than or equal to 5\;
    }
    \ForEach{y in 0..5} {
        $y \leftarrow y + 1$\;
    }
    \For{$y$ in $0..5$} {
        $y \leftarrow y - 1$\;
    }
    \While{$x > 5$} {
        $x \leftarrow x - 1$\;
    }
    \Return Return something here\;
    \caption{what}
\end{algorithm}

    \begin{problem}
        Calculate the integral of the function \( g(x) = 3x^2 \) with respect to \( x \).
        \end{problem}

        \begin{solution}
        To calculate the integral of \( g(x) = 3x^2 \), we use the power rule for integration:
        \[
        \int 3x^2 \, dx = x^3 + C
        \]
        where \( C \) is the constant of integration.
        \end{solution}

        \begin{problem}
        Determine whether the series \( \sum_{n=1}^{\infty} \frac{1}{n^2} \) converges or diverges.
        \end{problem}

        \begin{solution}
        The series \( \sum_{n=1}^{\infty} \frac{1}{n^2} \) is a \( p \)-series with \( p = 2 \). Since \( p > 1 \), the series converges by the \( p \)-series test.
        \end{solution}

        \begin{problem}
        Solve the equation \( 2x + 5 = 13 \).
        \end{problem}

        \begin{solution}
        Subtracting 5 from both sides, we get:
        \[
        2x = 8
        \]
        Dividing both sides by 2, we obtain:
        \[
        x = 4
        \]
        \end{solution}

\end{document}
