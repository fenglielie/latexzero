\documentclass{article}
%================================
% note-setup-leftsidebox.tex
% fenglielie@qq.com 2025-07-10
%================================

\usepackage{amsmath,amsthm,amsfonts,amssymb}
\usepackage{mathtools}
\usepackage{mathrsfs}
\usepackage{bm}
\usepackage{extarrows}
\usepackage[a4paper, margin=1in]{geometry}
\usepackage{float}
\usepackage{indentfirst}
\usepackage{anyfontsize}
\usepackage{booktabs,multirow,multicol}
\usepackage[shortlabels,inline]{enumitem}
\usepackage{appendix}

\usepackage[dvipsnames]{xcolor}
\usepackage{graphicx}
\graphicspath{
    {./figure/}{./figures/}{./image/}{./images/}{./graphic/}{./graphics/}{./picture/}{./pictures/}
}
\usepackage{subcaption}

\usepackage[ruled,linesnumbered,noline]{algorithm2e}
\usepackage{listings}
\lstdefinestyle{simpleStyle}{
    basicstyle=\ttfamily\small,
    breaklines=true,
    keywordstyle=\color{blue},
    identifierstyle=\color{black},
    stringstyle=\color{violet},
    commentstyle=\color[RGB]{34,139,34},
    showstringspaces=false,
    numbers=left,
    numbersep=2em,
    numberstyle=\footnotesize,
    frame=single,
    framesep=1em,
}
\lstset{style=simpleStyle}

\usepackage[colorlinks=true,linkcolor=,urlcolor=magenta,citecolor=violet]{hyperref}

\renewcommand*{\proofname}{\normalfont\bfseries Proof}

\usepackage{thmtools}

%% define environments

\declaretheorem[style=plain, name=Theorem, numbered=yes, numberwithin=section]{theorem}
\declaretheorem[style=plain, name=Theorem, numbered=no]{theorem*}

\declaretheorem[style=plain, name=Proposition, numbered=yes, sibling=theorem]{proposition}
\declaretheorem[style=plain, name=Proposition, numbered=no]{proposition*}

\declaretheorem[style=plain, name=Corollary, numbered=yes, sibling=theorem]{corollary}
\declaretheorem[style=plain, name=Corollary, numbered=no]{corollary*}

\declaretheorem[style=plain, name=Lemma, numbered=yes, sibling=theorem]{lemma}
\declaretheorem[style=plain, name=Lemma, numbered=no]{lemma*}

\declaretheorem[style=plain, name=Claim, numbered=yes, sibling=theorem]{claim}
\declaretheorem[style=plain, name=Claim, numbered=no]{claim*}

\declaretheorem[style=definition, name=Definition, numbered=yes, numberwithin=section]{definition}
\declaretheorem[style=definition, name=Definition, numbered=no]{definition*}

\declaretheorem[style=definition, name=Example, numbered=yes, numberwithin=section]{example}
\declaretheorem[style=definition, name=Example, numbered=no]{example*}

\declaretheorem[style=definition, name=Problem, numbered=yes, numberwithin=section]{problem}
\declaretheorem[style=definition, name=Problem, numbered=no]{problem*}

\declaretheorem[style=remark, name=Remark, numbered=yes, numberwithin=section]{remark}
\declaretheorem[style=remark, name=Remark, numbered=no]{remark*}

\declaretheorem[style=remark, name=Note, numbered=yes, numberwithin=section]{note}
\declaretheorem[style=remark, name=Note, numbered=no]{note*}

\declaretheoremstyle[headfont=\bfseries, bodyfont=\normalfont, spaceabove=3pt, spacebelow=3pt, qed=\ensuremath{\square}]{solutionstyle}

\declaretheorem[style=solutionstyle, name=Solution, numbered=yes, numberwithin=section]{solution}
\declaretheorem[style=solutionstyle, name=Solution, numbered=no]{solution*}

\usepackage[most]{tcolorbox}

\newcommand{\newtcbenvironment}[2]{
    \tcolorboxenvironment{#1}{#2, enhanced, breakable, sharp corners,leftrule=2pt, rightrule=0pt, toprule=0pt, bottomrule=0pt}
    \tcolorboxenvironment{#1*}{#2, enhanced, breakable, rounded corners,leftrule=2pt, rightrule=0pt, toprule=0pt, bottomrule=0pt}
}

%% define styles

\newtcbenvironment{theorem}{colframe=RoyalPurple, colback=RoyalPurple!8}
\newtcbenvironment{proposition}{colframe=RoyalPurple, colback=RoyalPurple!8}
\newtcbenvironment{corollary}{colframe=NavyBlue, colback=SkyBlue!8}
\newtcbenvironment{lemma}{colframe=NavyBlue, colback=SkyBlue!8}
\newtcbenvironment{claim}{colframe=NavyBlue, colback=SkyBlue!8}

\newtcbenvironment{definition}{colframe=ForestGreen, colback=ForestGreen!5}
\newtcbenvironment{example}{colframe=RawSienna, colback=RawSienna!5}
\newtcbenvironment{problem}{colframe=WildStrawberry!30, colback=WildStrawberry!5}


\title{\LaTeX{} Note Template}
\author{}
\date{}

\begin{document}

\section{Demo}

\begin{theorem}[xxx]
    If $1<p<\infty$ and $m > n/p$, or $p=1$ and $m \ge n$, there exist a constant $C = C(m,n,\gamma,p)$, such that
    \begin{equation*}
        \Vert R^m u \Vert_{L^\infty(\Omega)} \le C d^{m-n/p} |u|_{W^m_p(\Omega)}
    \end{equation*}
    for all $u \in W^m_p(\Omega)$.
\end{theorem}
\begin{proof}[\upshape\bfseries Proof of xxx]
    First, we assume that $u \in C^m(\Omega) \cap W^m_p(\Omega)$. We can use the pointwise representation of $R^mu(x)$.
    \begin{align*}
        |R^mu(x)| ={}  m \left| \sum_{|\alpha| = m} \int_{C_x} k_{\alpha}(x,z) D^\alpha u(z)\,dz \right|
        \le{}  C \sum_{|\alpha|=m} \int_{\Omega} |x-z|^{-n+m} |D^\alpha u(z)|\,dz
        \le{}       C' d^{m-\frac{n}{p}} |u|_{W^m_p(\Omega)}.
    \end{align*}
    The proof can be completed via a density argument.
\end{proof}

\begin{proposition}[xxx]
    \begin{equation*}
        Q^m u(x) = \sum_{|\lambda| < m} \left( \int_B \psi_\lambda(y) u(y)\,dy \right) x^\lambda
    \end{equation*}
    where $\psi_\lambda \in C_0^\infty(\mathbb{R}^n)$ and $\mathrm{supp}(\phi_\lambda) \in \overline{B}$.
\end{proposition}


\begin{corollary}[xxx]
    Under the assumption of xxx, the following inequality holds
    \begin{equation*}
        \inf_{v \in P^{m-1}} \Vert u - v \Vert_{W^k_p(\Omega)} \le C_{m,n,\gamma} d^{m-k} |u|_{W^k_p(\Omega)}, \,\, k = 0,1,\dots,m,
    \end{equation*}
\end{corollary}

\begin{lemma}[xxx]
    Let $f \in L^p(\Omega)$ for $p \ge 1$ and $m \ge 1$ and let $g(x) = \int_\Omega |x-z|^{-n+m} |f(z)|\,dz$.
    Then $\Vert g \Vert_{L^p(\Omega)} \le C_{m,n} d^m \Vert f\Vert_{L^p(\Omega)}$.
\end{lemma}

\begin{claim}[xxx]
    $Q^m u$ is a polynomial of degree less than $m$ in $x$.
\end{claim}


\begin{definition}[xxx]
    $\Omega$ is star-shaped with respect to the ball $B$ if , for all $x \in \Omega$, the closed convex hull of $\{x\} \cup B$ is a subset of $\Omega$.
\end{definition}


\begin{example}[xxx]
    The integral form of the Taylor remainder for $f \in C^m([0,1])$ is given by
    \begin{equation*}
        f(s) = \sum_{k=0}^{m-1}\frac{1}{k!} f^{(k)}(0) + \int_0^s \frac{1}{(m-1)!} f^{(m)}(t)(s-t)^{m-1}\,dt.
    \end{equation*}
\end{example}


\begin{problem}[xxx]
Calculate the integral of the function $g(x) = 3x^2$ with respect to $x$.
\end{problem}
\begin{solution}[xxx]
    To calculate the integral of $g(x) = 3x^2$, we use the power rule for integration:
    \[
        \int 3x^2 \, dx = x^3 + C
    \]
    where $C$ is the constant of integration.
\end{solution}


\begin{remark}[xxx]
    Such a polynomial is not unique, due to the choice od cut-off function $\phi$.
\end{remark}


\begin{note}[xxx]
    The degree of $Q^m u$ is at most $m-1$.
\end{note}

\end{document}
